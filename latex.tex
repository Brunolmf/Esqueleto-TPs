\newcommand{\RR}{\mathbb{R}}
\newcommand{\QQ}{\mathbb{Q}}
\newcommand{\CC}{\mathbb{C}}
\documentclass{article}
\usepackage[a4paper]{geometry}
\usepackage{graphicx}
\geometry{margin=2.5cm}

\title{\textbf{Estrutura de Dados - Trabalho Prático 2 }}
\author{
    \textbf{Bruno Lopes Melo Fonseca} \\
    Matrícula: 2023087869 \\
    \bigskip
    Universidade Federal de Minas Gerais (UFMG) \\
    Belo Horizonte - MG - Brasil \\
    \href{mailto:email@ufmg.br}{brunolmf@ufmg.br}
}
\date{}


\begin{document}

\maketitle

%%%%%%%%%%%%%%%%%%%%%%%%%%%%%%%%%%%%%%%%%%%%%%%%%%%%%%%%%%%%%%%%%%%%%%%%%%%%%%

\section{Introdução}

No campo da ciência da computação, o domínio sobre a implementação de diferentes estruturas de dados, suas vantagens, desvantagens e a complexidade de suas operações é fundamental para programadores. O objetivo deste trabalho é desenvolver uma simulação de eventos discretos para representar o funcionamento de um hospital, desde a admissão de pacientes até a geração de estatísticas sobre seu atendimento.

A solução proposta utiliza um \textit{min-heap} para gerenciar eventos, representando uma fila de prioridade, além de filas encadeadas para tratar os pacientes em diferentes estágios do processo hospitalar.

%%%%%%%%%%%%%%%%%%%%%%%%%%%%%%%%%%%%%%%%%%%%%%%%%%%%%%%%%%%%%%%%%%%%%%%%%%%%%%
\section{Implementação}


%%%%%%%%%%%%%%%%%%%%%%%%%%%%%%%%%%%%%%%%%%%%%%%%%%%%%%%%%%%%%%%%%%%%%%%%%%%%%%

\section{Análise Temporal}


\subsubsection{Classe Fila}

\subsubsection{Análise Geral}

%%%%%%%%%%%%%%%%%%%%%%%%%%%%%%%%%%%%%%%%%%%%%%%%%%%%%%%%%%%%%%%%%%%%%%%%%%%%%%

\section{Estratégias de robustez}

%%%%%%%%%%%%%%%%%%%%%%%%%%%%%%%%%%%%%%%%%%%%%%%%%%%%%%%%%%%%%%%%%%%%%%%%%%%%%%
\section{Análise experimental}

 \begin{figure}[H]
    \centering
    \includegraphics[width=0.6\textwidth]{regressao.jpeg}
    \caption{Custo do programa O(n)}
    \label{fig:minha-imagem}
\end{figure}
\vspace{1cm}



%%%%%%%%%%%%%%%%%%%%%%%%%%%%%%%%%%%%%%%%%%%%%%%%%%%%%%%%%%%%%%%%%%%%%%%%%%%%%%%%%%%%%%%%%%%%%%%

\section{Conclusão}


%%%%%%%%%%%%%%%%%%%%%%%%%%%%%%%%%%%%%%%%%%%%%%%%%%%%%%%%%%%%%%%%%%%%%%%%%%%%%%%%%%%%%%%%%%%%
\section {Bibliografia}

Ziviani, N. (2006). Projetos de algoritmos com implementações em Java e C++: Capítulo 3: Estruturas de dados básicas. São Paulo: Cengage Learning.
\par Livro Chaimowicz, L. and Prates, R. (2020). Slides virtuais da disciplina de estruturas de dados
disponibilizados via moodle. Departamento de Ciência da Computação. Universidade Federal de
Minas Gerais.

\par SUTTER, Herb. More Exceptional C++: 40 new engineering puzzles, programming problems, and solutions. Boston: Addison-Wesley, 2002.

